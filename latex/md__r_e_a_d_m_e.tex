\subsection*{概要}


\begin{DoxyItemize}
\item スタートモジュール用の\+Arduinoコード
\end{DoxyItemize}

\subsection*{ピン関係}

\tabulinesep=1mm
\begin{longtabu}spread 0pt [c]{*{2}{|X[-1]}|}
\hline
\PBS\centering \cellcolor{\tableheadbgcolor}\textbf{ ピン番号  }&\PBS\centering \cellcolor{\tableheadbgcolor}\textbf{ 役割   }\\\cline{1-2}
\endfirsthead
\hline
\endfoot
\hline
\PBS\centering \cellcolor{\tableheadbgcolor}\textbf{ ピン番号  }&\PBS\centering \cellcolor{\tableheadbgcolor}\textbf{ 役割   }\\\cline{1-2}
\endhead
3  &スタートモジュールのコース接触判定   \\\cline{1-2}
6  &D\+I\+Pスイッチbit0   \\\cline{1-2}
7  &D\+I\+Pスイッチbit1   \\\cline{1-2}
8  &D\+I\+Pスイッチbit2   \\\cline{1-2}
9  &D\+I\+Pスイッチbit3   \\\cline{1-2}
10  &P\+C通信用ソフトウェアシリアル\+RX   \\\cline{1-2}
11  &P\+C通信用ソフトウェアシリアル\+TX   \\\cline{1-2}
12  &スタートモジュールのコース通過判定   \\\cline{1-2}
A4(18)  &D-\/sub2番ピン(S\+DA)   \\\cline{1-2}
A5(19)  &D-\/sub3番ピン(S\+CL)   \\\cline{1-2}
G\+ND  &D-\/sub1番ピン   \\\cline{1-2}
\end{longtabu}



\begin{DoxyItemize}
\item 上記ピン番号について\+D-\/sub関係以外は仮決め
\begin{DoxyItemize}
\item というかコース接触判定・ゴール判定のピンを1個しか割り当ていないが、判定に複数ピンを使いたい場合もあるだろう…
\item そのときはコード修正必要
\end{DoxyItemize}
\item I2\+C関係ピンは固定
\item ソフトウェアシリアルのピン番号(\+R\+X, T\+X)はいくらでも変更可能
\end{DoxyItemize}

\subsection*{D-\/sub関係}


\begin{DoxyItemize}
\item スレーブのアドレス指定:スタートモジュールに近い順に1,2,3...
\item D-\/sub関係は\+I2C(2,3番ピン)のみに集約し、\+D-\/subの4,5番ピン(ゴール通知・コース接触通知)は不要となった
\item ただし、上記対応により各モジュールの通過判定が必須になった
\begin{DoxyItemize}
\item どのモジュールと通信すればよいかわからないと\+I2\+C通信できないため
\end{DoxyItemize}
\end{DoxyItemize}

\subsection*{I2\+C関係}


\begin{DoxyItemize}
\item I2\+Cの通信内容は以下の通り
\begin{DoxyItemize}
\item マスタからスレーブへ通信開始を通知する(1)
\item マスタからスレーブへコース接触、モジュール通過イベントの発生通知を要求する(2)
\end{DoxyItemize}
\item 詳細仕様
\begin{DoxyItemize}
\item 上記(1)についてはシンプル
\begin{DoxyItemize}
\item 図示すると以下のような流れ


\end{DoxyItemize}
\item 上記(2)についてはスレーブ→マスタへの通知になるため処理の流れが少し複雑になる
\begin{DoxyItemize}
\item 本来以下のような流れにできれば(1)同様シンプルになる


\item が、\+I2\+Cではスレーブからマスタに自発的に通信することはできない
\item そこで、以下のように実装した


\end{DoxyItemize}
\end{DoxyItemize}
\end{DoxyItemize}

\subsection*{処理の流れ}


\begin{DoxyEnumerate}
\item スタート待機状態
\end{DoxyEnumerate}
\begin{DoxyItemize}
\item P\+Cにスタートメッセージ(\textquotesingle{}s\textquotesingle{})を投げる
\item そのまま\+P\+Cから\textquotesingle{}s\textquotesingle{}が返ってきたら 2. スタートモジュール通過中状態 に遷移する
\item …ように今は実装してあるが、スタート検知できるようにするなら、以下のような変更が必要
\begin{DoxyItemize}
\item スタート検知したら 2. スタートモジュール通過中状態 に遷移するようにする
\end{DoxyItemize}
\end{DoxyItemize}
\begin{DoxyEnumerate}
\item スタートモジュール通過中状態
\end{DoxyEnumerate}
\begin{DoxyItemize}
\item コース接触を検知した場合
\begin{DoxyItemize}
\item P\+Cへ当たった判定メッセージ(\textquotesingle{}h\textquotesingle{})を投げる
\item そのまま\+P\+Cから\textquotesingle{}h\textquotesingle{}が返ってくるまで待機
\end{DoxyItemize}
\item モジュール通過を検知した場合
\begin{DoxyItemize}
\item P\+Cへ通過/ゴール判定メッセージ(\textquotesingle{}t\textquotesingle{})を投げる
\item そのまま\+P\+Cから\textquotesingle{}t\textquotesingle{}が返ってきたらスレーブ1との通信を開始し、 3. スレーブモジュール通過中状態 に遷移する
\end{DoxyItemize}
\end{DoxyItemize}
\begin{DoxyEnumerate}
\item スレーブモジュール通過中状態
\end{DoxyEnumerate}
\begin{DoxyItemize}
\item 一定時間ごとにスレーブモジュールの状態を確認する
\item スレーブモジュールの状態に応じて必要であれば\+P\+Cと通信を行う
\item 通信対象モジュールの切り替えも\+Dsub関係管理クラスの{\ttfamily handle\+\_\+dsub\+\_\+event()}の中で行われる 
\end{DoxyItemize}